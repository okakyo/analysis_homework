\synctex=1
\documentclass[dvipdfmx,10pt, a4j]{jarticle}
%----------------------------------------------------------
%パッケージ読み込み
\usepackage{amsmath}
\usepackage{amsthm}
\usepackage{amssymb}
\usepackage{amsthm} %定理環境の拡張
\usepackage{ascmac}
\usepackage{bm}
\usepackage{cases}
\usepackage{comment} %非表示にするためのコメント
\usepackage{enumerate}
\usepackage{float} %画像をその場に表示.[h]の代わりに[H]
\usepackage{graphicx} % eps 形式の図版取り込みのため
\usepackage{mathrsfs}
\usepackage{url}
\usepackage[dvipdfmx]{hyperref}
%----------------------------------------------------------

%----------------------------------------------------------
%命題関係の定義
\theoremstyle{definition} 
\newtheorem{definition}{定義}[section]
\newtheorem{theorem}{定理}[section]
\newtheorem{proposition}[theorem]{命題}
\newtheorem{lemma}[theorem]{補題}
\newtheorem{col}[theorem]{系}
\newtheorem{example}{例}[section]
\newtheorem{remark}{注意}[section]
\renewcommand*{\proofname}{証明}
%----------------------------------------------------------

%タイトル・著者===================================================
\title{第3回 レポート課題}
\author{岡 恭平}
%=================================================================

%本文開始=========================================================
\begin{document}

\maketitle

%カウンタ--------------------------------------------------
\setcounter{section}{1}
%\setcounter{subsection}{0}
%\setcounter{subsubsection}{0}
\setcounter{theorem}{49}
%----------------------------------------------------------

\begin{proposition}
教科書参照
\end{proposition}

\begin{proof}
  定義1.183 より
\begin{align*}
    \bar{\bm{x}} &=(\bar{\bm{x}_{1}},\bar{\bm{x}_{2}},{\cdots},\bar{\bm{x}_{n}})'
    \\
    &= \frac{1}{n}(\sum_{k=1}^{n}{\bm{x}_{k1}},\sum_{k=1}^{n}{\bm{x}_{k2}},{\cdots},\sum_{k=1}^{n}{\bm{x}_{kn}})'
    \\
    &= \frac{1}{n}(\sum_{k=1}^{n}{\bm{x}_{1}},\sum_{k=1}^{n}{\bm{x}_{2}},{\cdots},\sum_{k=1}^{n}{\bm{x}_{n }})
    \\
    &= \frac{1}{n}{\sum_{k=1}^{n}{\bm{x}}} = \frac{1}{n}\bm{X}'\bm{1}_{n}~~~~(\because 1.181)
  \end{align*}
  \begin{align*}
    \bm{{S}_{XX}}&=({s}_{{x}_{i}{x}_{j}})~~~~({i},{j}=1,2,3\cdots,{p})
    \\
    &=\frac{1}{n}\sum_{k=1}^{n}(\bm{x}_{i}-{\bar{\bm{x}}})(\bm{x}_{i}-{\bar{\bm{x}}})'
    \\
    &=\frac{1}{n}\sum_{k=1}^{n}(
      \bm{x}_{i}\bm{x}_{i}'
      -\bm{x}_{i}{\bar{\bm{x}}}'
      -{\bar{\bm{x}}}\bm{x}_{i}'
      +{\bar{\bm{x}}}{\bar{\bm{x}}'}
      )
    \\
    &=\frac{1}{n}\sum_{k=1}^{n}\bm{x}_{i}\bm{x}_{i}'-{\bar{\bm{x}}}{\bar{\bm{x}}}'~~~~(\because 1.101)
    \\
    &=\frac{1}{n}\bm{X}'\bm{X}-{\bar{\bm{x}}}{\bar{\bm{x}}}'~~~~(\because 行列の積)
    \\
    &=\frac{1}{n}\bm{X}'\bm{X}- (\frac{1}{n}\bm{X}'\bm{1}_{n})(\frac{1}{n}\bm{1}_{n}'\bm{X})
    \\
    &=\frac{1}{n}(\bm{X}'\bm{X}-{\frac{1}{n}\bm{X}'\bm{1}_{n}\bm{1}_{n}'\bm{X}})
    \\
  \end{align*}
  以上と同様にして、$\bm{S}_{xy}、\bm{S}_{XY}$でも成立する。
\end{proof}

\begin{proposition}
教科書参照
\end{proposition}

\begin{proof}

\begin{align*}
  \bm{R}_{XX} &= (\bm{r}_{{x}_{i}{x}_{j}})~~~({i},{j}=1,2...p)
  \\
  &=
\end{align*}
   
\end{proof}

\begin{proposition}
  教科書参照
\end{proposition}
% ---

\begin{proof}
$[\bm{A},\bm{B}]$の列空間への射影行列は
\begin{align*}
    [\bm{A},\bm{B}]([\bm{A},\bm{B}]^{\top}[\bm{A},\bm{B}])^{-1}[\bm{A},\bm{B}]^{\top}
    &= 
    [\bm{A},\bm{B}]
    \begin{bmatrix}
        \bm{A}^{\top}\bm{A} & \bm{A}^{\top}\bm{B} \\
        \bm{B}^{\top}\bm{A} & \bm{B}^{\top}\bm{B} \\
    \end{bmatrix}^{-1}
    [\bm{A},\bm{B}]^{\top}
    \\
    &= 
    [\bm{A},\bm{B}]
    \begin{bmatrix}
        \bm{A}^{\top}\bm{A} & \bm{O} \\
        \bm{O} & \bm{B}^{\top}\bm{B} \\
    \end{bmatrix}^{-1}
    [\bm{A},\bm{B}]^{\top}
    \\
    &= 
    [\bm{A},\bm{B}]
    \begin{bmatrix}
        (\bm{A}^{\top}\bm{A})^{-1} & \bm{O} \\
        \bm{O} & (\bm{B}^{\top}\bm{B})^{-1} \\
    \end{bmatrix}
    [\bm{A},\bm{B}]^{\top}
    \\
    &= 
    \begin{bmatrix}
        \bm{A}(\bm{A}^{\top}\bm{A})^{-1} ,&
        \bm{B}(\bm{B}^{\top}\bm{B})^{-1} \\
    \end{bmatrix}
    \begin{bmatrix}
        \bm{A}^{\top} \\
        \bm{B}^{\top} \\
    \end{bmatrix}
    \\
    &= \bm{A}(\bm{A}^{\top}\bm{A})^{-1}\bm{A}^{\top} + \bm{B}(\bm{B}^{\top}\bm{B})^{-1}\bm{B}^{\top}
\end{align*}
となる.

一方で,$\bm{A}$の列空間への射影$\bm{\bm{P}}_{\bm{A}}$を考えると,
\begin{align*}
    \bm{P}_{\bm{A}} = \bm{A}(\bm{A}^{\top}\bm{A})^{-1}\bm{A}^{\top}
\end{align*}
であり,$\bm{A}$の列空間と直交する空間への射影を$\bm{P}^{\bot}_{\bm{A}}$とすると
\begin{align*}
    \bm{P}_{\bm{A}} + \bm{P}^{\bot}_{\bm{A}} = I
\end{align*}
が成り立つ.
いま$\bm{A}^{\top}\bm{B}=\bm{O}$より,$\bm{B}$の列空間と$\bm{A}$の列空間が直交することから,
\begin{align*}
    \bm{P}_{\bm{A}} + \bm{P}_{\bm{B}} = I
\end{align*}
が成り立つ.

よって,
$\bm{A}(\bm{A}^{\top}\bm{A})^{-1}\bm{A}^{\top} + \bm{B}(\bm{B}^{\top}\bm{B})^{-1}\bm{B}^{\top} = I$が成り立つ.

\end{proof}

\end{document}
%本文ここまで=========================================================
